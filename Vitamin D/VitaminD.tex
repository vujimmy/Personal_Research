\documentclass{article}
\usepackage{graphicx} % Required for inserting images
\usepackage[backend=bibtex]{biblatex}
\addbibresource{references.bib} %Import the bibliography file
\usepackage{hyperref}
\usepackage{xcolor}
\usepackage{float}
\usepackage{geometry}
\author{Jimmy}
\geometry{a4paper, margin=1in}
\title{\textbf{Vitamin D}}
\date{Infos assembled by Jimmy \\
All the references are displayed at the end of the document
Started writing on 18th November 2025
}
\begin{document}


\maketitle

\newpage
\tableofcontents
\newpage

\section{What are vitamins, in general ?}
Simply put, vitamins are vital substances for the well-being of a person. 
They are active in growth, skeletal development, use of micro-nutrients (like Calcium, Zinc, Magnesium, Phosphorus etc.), sight (e.g. Vitamin A), blood clotting, nervous system and immune system, DNA production, etc. \cite{ANSESWHATAREVITAMINS}. 
However, the body cannot synthetize vitamins by itself, EXCEPT Vitamin D which is the main focus of this paper, and vitamin K.
An appropriate intake of vitamins can prevent cancer, cardiovascular diseases and many age-related diseases.
Overconsumption of vitamins can be toxic for the body and vitamin deficiency can lead to clinical or pathological disorders.
\\
Vitamins DO NOT give energy as they contain 0 calorie, thus "faire le plein de vitamines pour avoir de l'énergie" is a misleading sentence, but having an adequate intake of all the needed vitamins is good to perform your daily activities and to keep your body healthy and functional.



\section{What is Vitamin D ?}
\label{What is Vitamin D ?}
There are two types of Vitamin D (but they are very very similar nutritionally, so we group them as just "Vitamin D") : Vitamin D2 and Vitamin D3.
\begin{itemize}
    \item Vitamin D2 is found in plants and shrooms (pretty vegan friendly), 
    \item Vitamin D3 is mostly found in oily fish, egg yolk and fortified diary products (a lot less vegan friendly).
\end{itemize}
Vitamin D is a liposoluble vitamin, which means it dissolves in fat \cite{ANSESWHATAREVITAMINS}. 
Because Vitamin D is stored in fat, there is a higher risk of toxicity if the intake is too high. 
The amount of fat with which vitamin D is ingested does not seem to significantly modify the bioavailability (aka absorption) of vitamin D3 \cite{borelVitaminBioavailabilityState2015}. \\
If the body is exposed to the sunlight, it can synthetize this vitamin (more precisely Vitamin D3). 
15 to 20 minutes in the sun (safely) is enough to get your daily intake of Vitamin D, as sunlight is the major source of this vitamin.
Dark skin people can have a harder time synthetizing Vitamin D and sunscreen can also impair its synthesis.
The body stops producing vitamin D from sunlight when the needs are met, so over-exposition to sunlight for Vitamin D is useless (or even dangerous, c.f. skin cancer) \cite{VeganPratique}. 
\\ 
Vitamin D plays a crucial role in the quality of bone tissues, muscular tissues and immunue system, as almost every organ and cell in the body has a Vitamin D receptor \cite{holickEvidencebasedDbateHealth2012}.




\section{Vitamin D benefits}
\subsection{Calcium and Phosphorus regulation}
\subsubsection{Calcium}
Vitamin D helps in abosorbing Calcium (which is not that easy to get) \cite{ANSESWHATAREVITAMINS} \cite{grantBenefitsRequirementsVitamin} and Phosphorus in the blood. 
\\
A good Calcium regulation leads to : 
\begin{itemize} 
    \item optimal tissue mineralization : bones, teeth and cartilage (ears, nose),
    \item good musucular contraction,
    \item good nervous transmission,
    \item good coagulation (liquid blood changes into semisolid blood clots, which helps preventing blood loss from damaged blood vessels) \cite{BloodCoagulation}
\end{itemize}
Too much Vitamin D tends to increase Calcium levels too much in the blood, which is NOT good for cardiovascular and renal health. 
\subsubsection{Phosphorus}

Phosphorus is a very promiment component in the body and is fairly easy to find in food. 
It helps with apoptosis (the natural and planned death of the cells) and 
it is a component of 
ATP (Adenosine Triphosphate, that comes often in nutrition), which acts as the body primary source of energy. 
\\
Phosphorus is also found in DNA (which stores genetic information) and RNA (a messenger that helps to synthetize proteins). 
Thus, without phosphorus, the body would NOT be able to properly creates genes, proteins and new cells.
More, phospholipids (phosphorus + saturated fat + unsaturated fat) make up your cell membranes \cite{Phosphorus}.
\\
% Therefore, Vitamin D is very useful to ease the absorption of two essential nutrients.
\subsection{Slowing down cell aging}
A very recent study (June 2025) \cite{zhuVitaminD3Marine2025} suggests that Vitamin D supplements may lead to a strategy to counter biological aging.
Telomeres are regions of DNA \cite{Telomere} and shorten every time a cell divides; this shortening has been linked to aging and to age-related diseases like vascular dementia, type II diabetes, and cancer.
The study showed that the telomeres exposed to Vitamin D are longer than those who were not, which is equivalent to 3 years of aging \cite{NHI}.
\section{Vitamin D deficiency}
Vitamin D deficiency is very common worldwide (1 billion children and adults at risk \cite{holickHealthBenefitsVitamin2011}). 
% Furthermore, aging decreases the body's ability to absorb Vitamin D. 
\\
Vitamin D deficiency can lead to :
\begin{itemize}
    \item Muscular issues (weakness, aches) \cite{ANSESWHATAREVITAMINS} \cite{holickHealthBenefitsVitamin2011} \cite{grantBenefitsRequirementsVitamin}
    \item Osteomalacia (decalcification of the bones, which can lead to bone deformation and a higher risk of fracture) \cite{Osteomalacie},
    \item Osteoporosis (your bones become weak and brittle) \cite{ANSESWHATAREVITAMINS}
    \item Increase risk of cardiovascular diseases \cite{holickHealthBenefitsVitamin2011}
    \item Increase risk of autoimmune diseases \cite{holickHealthBenefitsVitamin2011}
    \item Type 1 and Type 2 diabetes \cite{grantBenefitsRequirementsVitamin} \cite{zittermannEstimatedBenefitsVitamin2010} \cite{holickHealthBenefitsVitamin2011}
    \item Anemia (abnormal low level of hemoglobin, a substance which helps red blood vessels to carry oxygen to all organs)\cite{ANSESWHATAREVITAMINS} 
    \item Alzheimer's disease \cite{holickHealthBenefitsVitamin2011}
    \item Breast, colon, pancreas and prostate cancer \cite{krishnanPotentialTherapeuticBenefits2012}\cite{holickHealthBenefitsVitamin2011}
\end{itemize}
Furthermore, pregnant and lactating women need more Vitamin D than usual, as biochemical disturbances and bone issues can have occur in the infant \cite{grantBenefitsRequirementsVitamin}.
A study from Karolina Lagowska \cite{lagowskaRelationshipVitaminStatus2018} reported that low vitamin D concentrations co-occur with disturbed menstrual cycles : women who did not meet the recommended level of Vitamin D had almost five times the odds of having menstrual cycle disorders as women who were above the recommended vitamin D level.

\section{Where to find Vitamin D ?}
The main source of Vitamin D is the sunlight. 
Here are good sources of Vitamin D \cite{ANSES_VITAMIND} \cite{pharmaciepolygone}:
\begin{itemize}
    \item Egg yolks
    \item Oily fish (poissons gras, e.g. salmon, tuna, anchovy, etc.)
    \item Some shrooms (e.g. girolles, cèpes etc.)
    \item Black chocolate
    \item Butter
    \item Vitamin D fortified products (e.g milk or cereals)
    \item Vitamin D complementation
\end{itemize}
It is possible to get the daily intake of Vitamin D while following a vegan diet, as Vitamin D3 is produced by sun exposition and Vitamin D2 is mostly found in plants, c.f. Section \ref{What is Vitamin D ?}. \\
Fun fact : vitamin D food fortification is the lowest in France \cite{ovesenGeographicalDifferencesVitamin2003}.

\section{Conclusion}
As Vitamin D is used almost everywhere in our body, it is important to have an adequate intake of this vitamin. 
Taking advantage of our ability to synthetize it easily, going out in the sunlight every day is an excellent way to meet our daily Vitamin D intake. 
However, it is also one of the most common deficiency, thus Vitamin D rich food and complementation may help to solve this issue, while being careful as to not consume it in excess.
\newpage
\printbibliography
\end{document}