\documentclass{article}
\usepackage{graphicx} % Required for inserting images
\usepackage[backend=bibtex]{biblatex}
\addbibresource{references.bib} %Import the bibliography file
\usepackage{hyperref}
\usepackage{xcolor}
\usepackage{float}
\usepackage{geometry}
\author{Jimmy}
\geometry{a4paper, margin=1in}
\title{\textbf{Vitamin D}}
\date{Infos assembled by Jimmy \\
All the references are displayed at the end of the document
}
\begin{document}


\maketitle

\newpage
\tableofcontents
\newpage

\section{What are vitamins, in general ?}
Simply put, vitamins are vital substances for the well-being of a person. 
They are active in growth, skeletal development, use of micro-nutrients (like Calcium, Zinc, Magnesium etc.), sight (e.g. Vitamin A), blood clotting, nervous and immune system, DNA production, etc. \cite{ANSESWHATAREVITAMINS}. 
However, the body cannot synthetize vitamins by itself, except Vitamin D which is main focus, and vitamin K.
An appropriate intake of vitamins can prevent cancer, cardiovascular diseases and many age-related diseases.
Overconsumption of vitamins can be toxic for the body and vitamin deficiency can lead to clinical or pathological disorders.
Vitamins DO NOT give energy, they contain 0 calorie, thus "faire le plein de vitamines pour avoir de l'énergie" is a misleading sentence.






\section{What is Vitamin D ?}
Vitamin D is a liposoluble vitamin, which means it dissolves in fat \cite{ANSESWHATAREVITAMINS}. 
% Because Vitamin D is stored in fat, there is a higher risk of toxicity if the intake is too high. 
\\
If the body is exposed to the sunlight, it can synthetize this vitamin. 
15 to 20 minutes in the sun (safely) is enough to get your daily intake of Vitamin D, as sunlight is the major source of this vitamin.
\\ 
Vitamin D plays a crucial role in the quality of bone tissues, muscular tissues and immunue system, as almost every organ and cell in the body has a Vitamin D receptor \cite{holickEvidencebasedDbateHealth2012}.







\section{Calcium and Phosphorus regulation}
Vitamin D helps in abosorbing Calcium \cite{ANSESWHATAREVITAMINS} \cite{grantBenefitsRequirementsVitamin} and Phosphorus in the blood. 
\\
A good Calcium regulation leads to : 
\begin{itemize} 
    \item optimal tissue mineralization : bones, teeth and cartilage (ears, nose),
    \item good musucular contraction,
    \item good nervous transmission,
    \item good coagulation
\end{itemize}




\section{Vitamin D deficiency}
Aging decreases the body's ability to absorb Vitamin D.
Vitamin D deficiency is very common worldwide (1 billion children and adults at risk \cite{holickHealthBenefitsVitamin2011}) and can lead to :
\begin{itemize}
    \item Muscular issues (weakness, aches) \cite{ANSESWHATAREVITAMINS} \cite{holickHealthBenefitsVitamin2011} \cite{grantBenefitsRequirementsVitamin}
    \item Osteomalacia (decalcification of the bones, which can lead to bone deformation and a higher risk of fracture) \cite{Osteomalacie},
    \item Osteoporosis (your bones become weak and brittle) \cite{ANSESWHATAREVITAMINS}
    \item Increase risk of cardiovascular diseases \cite{holickHealthBenefitsVitamin2011}
    \item Increase risk of autoimmune diseases \cite{holickHealthBenefitsVitamin2011}
    \item Type 1 and Type 2 diabetes \cite{grantBenefitsRequirementsVitamin} \cite{zittermannEstimatedBenefitsVitamin2010} \cite{holickHealthBenefitsVitamin2011}
    \item Anemia \cite{ANSESWHATAREVITAMINS}
    \item Alzheimer's disease \cite{holickHealthBenefitsVitamin2011}
    \item Breast, colon, pancreas and prostate cancer \cite{krishnanPotentialTherapeuticBenefits2012}\cite{holickHealthBenefitsVitamin2011}
\end{itemize}
Furthermore, pregnant and lactating women need more Vitamin D than usual, as biochemical disturbances and bone issues can have occur in the infant \cite{grantBenefitsRequirementsVitamin}.
A study from Karolina Lagowska \cite{lagowskaRelationshipVitaminStatus2018} reported that low vitamin D concentrations co-occur with disturbed menstrual cycles : women who did not meet the recommended level of Vitamin D had almost five times the odds of having menstrual cycle disorders as women who were above the recommended vitamin D level.
Vitamin D food fortification is the lowest in France \cite{ovesenGeographicalDifferencesVitamin2003}

\printbibliography
\end{document}