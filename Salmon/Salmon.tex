\documentclass{article}
\usepackage{graphicx} % Required for inserting images
\usepackage{biblatex}
\addbibresource{references.bib} %Import the bibliography file
\usepackage{hyperref}
\usepackage{xcolor}
\usepackage{float}
\usepackage{geometry}
\geometry{a4paper, margin=1in}
\title{\textbf{Saumon (Norvège, Ecosse, Alaska)}}

\date{Infos assembled by Dim \\
Most of the document is C/C from various sources\\
All the references are displayed at the end of the document
}
\begin{document}


\maketitle

\newpage
\tableofcontents
\newpage

\section{Salmon}
En plat principal chez soi ou à la cantine, en sushis, pour les fêtes de fin d'année, le saumon est devenu incontournable dans notre alimentation.
Il est le poisson le plus consommé en France, sa production mondiale a grimpé de $70\%$ entre 2010 et 2018. $98\%$ du saumon que l'on consomme est du poisson d'élevage et les Français sont les plus gros consomamteurs de saumon en Europe \cite{FT3}. De plus, la France est le premier importateur de saumon d'Ecosse. 
La Norvege est le territoire de prédilection du saumon de l'Atlantique.
En Norvege, $90 \%$ du saumon est issu d'élevage, la saison du saumon sauvage ne dure que 2 mois par an et il est plus cher. Parfois, les cours d'eau où se trouve le saumon savage sont fermés pour lui permettre de se reproduire tranquillement et ainsi régénerer le stock, mais cela contribue à la rareté de ce poisson et donc à une hausse du prix de vente.

Le saumon sauvage se nourrit de crevettes, de krills et d'autres crustacés et ce régime lui donne sa couleur rouge-orange.
Le saumon d'élevage est nourri par du krill articifiel pour obtenir la teinte souhaitée (sinon le saumon ne se vend pas). Il y a une même "échelle", un "éventail de couleurs" qui s'appelle un "salmofan" inventé par Hoffman-La-Roche (une entreprise pharmaceutique) et qui permet de quantifier la couleur d'un saumon.

En s'échappant des fermes piscioles, les poissons d'élevage peuvent se reproduire avec les poissons sauvages, ce qui perturbe le patrimoine génétique du poisson sauvage
Egalement, les granulés utilisés dans l'élevage contiennent des additifs pour donner la "bonne couleur" au samon et peuvent être ingérés par les poissons sauvages. Par ailleurs, les poissons sauvages sont carnivores tandis que les poissons d'élevage sont herbivores parce que c'est moins cher.



\section{Salmon benefits}
Source d'omega 3, de protéines etc.

\section{Menaces contre les saumons}
Les centrales hydrauliques, qui font varier constamment le niveau de l'eau et rend plus difficile l'accès aux frayères pour les saumons sauvages.
Le réchauffement des mers
Les fermes piscicoles qui rejettent des déchets nocifs dans les eaux marines et le poux de mer qui a proliféré de façon incontrôlée

\subsection{Le poux de mer}
Le poux de mer est un crustacé parasite qui se nourrit de la peau et du sang de son hôte et peut ravager les autres poissons, en plus des saumons : des truites ont été retrouvées avec la peau et les nageoires rongées par le parasite. Les poissons sont donc grignotés vivants par cet animal.
Près de $100\%$ des poissons d'élevage sont attaqués par le poux pendant l'année. SOURCE ?
Sa proflifération hors de contrôle coûte des millions d'euros à l'industrie de la pêche. Le réchauffement de l'eau encourage également la profilération du parasite : le nombre de poux de mer a doublé en 2024.



Les poissons qui s'échappent

Conséquences : moins de saumons sauvages à pêcher


Pour contrer les poux de mer, certains élevages utilisent des systèmes fermés comme des bassins étanches qui en plus ne rejettent pas de déchets dans la mer (ils peuvent être recyclées ou séchés pour faire de l'engrais).
Les scientifiques sont en train de développer un vaccin contre le poux de mer à injecter au saumon et recommandent au gouvernement des restrictions d'élevage : si $30 \%$ du saumon meurt dans une ferme à cause des poux de mer alors l'élevage doit réduire sa production de $6 \%$.
Les restrictions ont un impact économique assez important sur ce secteur, les aquaculteurs sont généralement très mécontents lorsque ce type de restriction est actif.
Des médicaments, des pesticides et des antibiotiques \cite{FT2} \cite{FT3} sont parfois également utilisés (et rejetés dans l'eau) pour lutter contre le poux de mer. 
D'autres éleveurs utilisent des solutions naturelles comme du labre ou du lompe, des poissons nettoyeurs qui mange les poux de mer, pour protéger leurs saumons.
\section{La Norvege}
La Norvege est le $1^{er}$ producteur de saumon de la planète avec 1M de tonnes d'exportation dans le monde

\section{Elevage intensif du saumon}
Les saumons d'élevage intensif sont beaucoup entassés dans les bassins. Parfois, certains sont complètement rongés par les poux de mer, si bien que leur chair est visible sous la forme d'une "plaque rose".
Les parasites sont dans des conditions idéales pour se multiplier de manière exponentielle \cite{FRANCE24}. Le saumon est même qualifié de "poulet en batterie de la mer".
En moyenne, le saumon d'élevage est 3 fois plus gras que le saumon sauvage \cite{FT2}. Des déchets plastiques, des saumons morts, des tâches d'essence et des excréments ont été retrouvés dans une aquaculture d'Ecosse \cite{FT2}

Les saumons sont nourris avec de la farine et de l'huile, qui sont eux-même produits avec des tonnes, des tonnes et encore des tonnes de poissons sauvages de Mauritanie \cite{ARTE2}, par exemple de la sardinelle ronde.
Pour produire 1kg de farine, il faut entre 4 et 5kg de poissons frais. Les usines ont interdiction de produire plus de 2000 tonnes de farine par an sous peine de sanction, mais la limitation n'est pas respectée : certaines produisent même plus de 30 000 tonnes de farines par an.
$23 \%$ de cette farine part en Europe, en particulier chez une entreprise française qui s'appelle "Olvea" qui ne publie pas ses comptes depuis 2018 ce qui est illégal.
Les huiles produites sont intégrés aux granulés qui nourissent les saumons d'élevage, qui sont ensuite vendus dans les grandes surfaces.
Des rejets dans la mer de substances plus que suspectes sont à signaler comme des liquides mousseux, fumants ou "chauds" (48.3° d'après un rapport de L'IMROP). De plus, (vraiment) énormément de poissons morts sont échoués en contrebas des usines.

Le label "Agriculture biologique" donne des garanties sur la qualité de l'eau ou la densité des poissons dans la ferme, mais ne donne aucune garantie sur l'absence de poissons sauvages dans l'alimentation.
Même, le saumon bio est le saumon qui engloutit le plus de poissons sauvages : il est nourrit jusqu'à $70\%$ de farine et d'huile de poisson, contre $20\%$ pour les poissons non bio. Donc 1kg de saumon bio aura nécessité 4kg de poisson sauvage.

\section{Saumons sauvages}
Alaska, saumon argenté. Il y a beaucoup de saumons sauvages en Alaska.
D'après le reportage \cite{FT1}, lorsqu'ils veulent se reproduire, les poissons sauvages d'Alaska remontent la rivière et sont piégés dans une écloserie industrielle. On coupe le ventre des femelles pour récupérer les oeufs et on récupère le sperme des mâles pour féconder les oeufs. Ensuite, les oeufs sont envoyés dans des bacs
où ils attendent d'éclore. Le taux de survie en écloserie atteint les $95\%$, ce qui est beaucoup plus élevé que dans la nature. Lorsque les oeufs éclosent, les nouveaux poissons sont mis dans des bassins et nourris avec des granulés contenant de la farine et de l'huile provenant de poissons pêchés au Chili.
Lorsque les saumons deviennent assez gros, ils sont relâchés dans la nature, dans l'océan Pacifique. Puis ils reviennent lorsqu'ils veulent se reproduire et le cycle recommence. Ce processus a fait grimper le nombre de saumons en Alaska de 22M en 2001 à 79M en 2021. Comme le nombre a énormément augmenté, il y a moins de nourriture pour chaque saumon, ce qui fait que chaque saumon est naturellement plus petit.

\printbibliography
\end{document}
